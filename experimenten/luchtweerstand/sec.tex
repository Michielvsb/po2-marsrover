\subsection{Luchtweerstand}
De luchtweerstand is naast de rolweerstand een andere oorzaak van 
energieverlies. Deze afwijking moet in rekening gebracht worden bij het 
opstellen van de bewegingsvergelijking van de wagen.

De wagen wordt losgelaten van een bepaalde hoogte op een hellend vlak. De wagen 
wordt hierbij gefilmd van bovenaf. Met behulp van video-analyse  kan de 
theoretisch opgestelde baan vergeleken worden met de experimenteel afgelegde 
baan. De theoretische baan houdt alle tot dan toe gekende parameters in 
rekening. Door een variabele luchtweerstandcoëfficient in te voeren kan de beste 
benadering tot de praktische baan gevonden worden.

De luchtweerstand blijkt minder belangrijk dan de rolweerstand. Enkel op hogere 
snelheden, die de wagen niet bereikt, speelt de luchtweerstand een significante 
rol.
