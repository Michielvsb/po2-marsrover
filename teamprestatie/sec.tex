\section{Teamefficiëntie en deadlines}
\label{section:teamefficientie}

De oorspronkelijke planning bleek te optimistisch. Vooral op het vlak van de bouw van het wagentje en het elektrisch schema bleek veel meer tijd nodig.
Bij de bouw van het mechanisch gedeelte waren er enkele problemen die naar boven kwamen. Zo bleek het karretje te groot om efficiënt te navigeren.
Een tweede versie van het karretje had een kleiner grondvlak, maar omwille van de keuze voor twee motoren, moest de plaat nog steeds relatief groot zijn.
Verder heeft het erg lang geduurd vooraleer de elektrische schakeling klaar was. Het gevolg hiervan was dat de programmacode slechts op het allerlaatste moment getest kon worden  met de hardware. Hierdoor was het wagentje niet klaar om de demonstratie te doen.

Een meer planmatige aanpak zou deze problemen beter kunnen voorkomen. Ook het strikter naleven van de deadlines had veel kunnen helpen.

Door het laattijdig afwerken van het mechanisch en het elektronisch gedeelte, zijn ook de experimenten naar latere tijdstippen verplaatst. De aanpak van de experimenten was meer planmatig.
De planning van de experimenten en het opstellen van de Maplebestanden gebeurde in afwachting van het mechanisch gedeelte. Ook het maken van het animatiefilmpje gebeurde in die periode.