
\section{Resultaten demonstratie}
De marswagen is in mate van het mogelijke in orde gebracht voor de demonstratie. 
De servomotor van het radarsysteem, die tijdens enkele tests het begaf, is 
vervangen door een andere servomotor. De software echter is echter niet binnen 
de overige beschikbare tijd aangepast kunnen worden. Daarom wordt slechts een 
beperkt onderdeel van de gehele code geactiveerd. De marswagen zal hierdoor 
enkel in staat zijn rechtdoor te rijden en bij het tegenkomen van de muur zijn 
eindsignaal geven.
Naar aanvang van de demonstratie, wordt het wagentje geplaatst op het parcours. 
De wagen begint te rijden na het induwen van de reset knop. De wagen versnelt 
een korte tijd, waarna de motoren worden uitgeschakeld. Deze onderbreking dient 
als een snelheidsbeperking, dit geeft de tijd die nodig is voor het bijsturen 
van het wagentje. Echter bijsturen doet het niet, aangezien het draaien van de 
radar is uitgeschakeld. Zo komt het wagentje nog voor het einde van het eerste 
rechte stuk tot stilstand. Het wijkt lichtjes af waardoor de afstandsensor, die 
voorwaarts gericht staat, de linkerzijmuur waarneemt en het wagentje tot 
stilstand wordt gebracht. Ingesteld als eindsignaal, begint een van de LEDjes 
van de Arduino Uno-controller te branden. Dit wordt echter niet bemerkt door de 
assistenten en de jury, omdat dit maar een klein, onopvallend signaal is en er 
een GPIO kabel het zicht blokkeerde.
De marsauto heeft het parcours niet helemaal afgelegd zoals verwacht. De wagen 
haalde het einde van het eerste rechte stuk niet, dit door een lichte afwijking 
die niet kan worden gecorrigeerd. Wel geeft het een eindsignaal bij het bemerken 
van een muur op bepaalde afstand aan de voorzijde van het wagentje. Ook de 
snelheid die het wagentje maximaal bereikt, is laag genoeg om manoeuvres uit te 
voeren.