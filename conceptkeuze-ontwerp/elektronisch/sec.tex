 
\subsection{Elektronisch}
De aansturing van de Mars Rover gebeurt via een Arduino Uno-controller. Deze 
microcontroller kan een analoog of digitaal signaal versturen naar het 
elektronisch gedeelte van het wagentje. Voor het realiseren van een autonome 
rover staan verschillende sensoren en actuatoren op de wagen gemonteerd die 
onderling met elkaar communiceren via de controller.

Het meten van de omgeving gebeurt via een optische afstandssensor, gemonteerd op 
een servomotor. Die heeft een bereik van 0\degree~tot 180\degree~en richt de afstandssensor 
beurtelings vooruit en zijdelings. Op die manier kan de omgeving in drie 
richtingen in kaart gebracht worden. De configuratie van een servomotor met een 
afstandssensor is goedkoper dan drie afstandssensoren die in de te verkennen 
richtingen gefixeerd worden, maar heeft als nadeel dat de afstand tot obstakels 
niet continu in elke richting bepaald kan worden.

De afstandssensor bestaat uit een infraroodzender en een lichtsensor. De sensor 
meet het tijdsverschil tussen het verzenden van het licht en het moment wanneer 
de lichtsensor een gereflecteerd signaal terug ontvangt en bepaalt zo de afstand 
tot een obstakel.\cite{SharpDistanceSensor} De keuze viel op deze sensor omdat die de meest bruikbare 
informatie geeft van alle sensors. Met een lichtsensor is het lastig werken 
zonder lichtbronnen en als er gewerkt wordt met geleiding (de aluminiumstrip) 
bestaat de output enkel uit ‘ja of neen’, zonder verdere informatie over de 
plaats van de rover in zijn omgeving.

Aan het einde van het parcours zal de afstandssensor zowel vooraan, links als 
rechts het signaal krijgen dat er een obstakel in de weg staat. Dan zal er een 
LED-licht knipperen.

Om de rover aan te drijven wordt een MM28 DC-motor gebruikt. Het inbrengen van 
een elektronische schakelaar, namelijk een relais, realiseert de ompoling van de 
motor. Zo is de rover in staat om in beide richtingen te kunnen rijden. Ook de 
snelheid kan gevarieerd worden door het gebruik van een MOSFET-transistor zodat 
de rover in de bochten zijn snelheid kan vertragen.

Voor de besturing van de rover wordt één servomotor gebruikt die aangesloten is 
op een stuurmechanisme. Aan de hand van een stuursignaal, gebaseerd op 
pulsbreedtes, kan de servomotor in de gewenste hoek geplaatst worden.

Aangezien er geen grote stromen door de Arduino-controller mogen lopen (maximaal 
grootteorde van enkele tientallen mA), moeten de signalen en de eigenlijke 
stroomtoevoer voor de motoren gesplitst worden. Daarom wordt gebruik gemaakt van 
een MOSFET-transistor. Dit soort transistor sluit de stroomkring met de batterij 
en de motoren van zodra de drempelspanning overschreden wordt. Deze 
drempelspanning wordt verkregen tussen gate- en sourcepin van de MOSFET.\cite{MosfetBitwizard} 
