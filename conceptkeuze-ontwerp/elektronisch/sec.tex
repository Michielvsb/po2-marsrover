 
\subsection{Elektronisch}
De aansturing van de Mars Rover gebeurt via een Arduino Uno-controller. Deze 
microcontroller kan een analoog of digitaal signaal versturen naar het 
elektronisch gedeelte van het wagentje. Voor het realiseren van een autonome 
rover staan verschillende sensoren en actuatoren op de wagen gemonteerd die 
onderling met elkaar communiceren via de controller.

Het meten van de omgeving gebeurt via een optische afstandssensor, gemonteerd op 
een servomotor. Die heeft een bereik van 0\degree~tot 180\degree~en richt de afstandssensor 
beurtelings vooruit en zijdelings. Op die manier kan de omgeving in drie 
richtingen in kaart gebracht worden. De configuratie van een servomotor met een 
afstandssensor is goedkoper dan drie afstandssensoren die in de te verkennen 
richtingen gefixeerd worden, maar heeft als nadeel dat de afstand tot obstakels 
niet continu in elke richting bepaald kan worden.

De afstandssensor bestaat uit een infraroodzender en een lichtsensor. De sensor 
meet het tijdsverschil tussen het verzenden van het licht en het moment wanneer 
de lichtsensor een gereflecteerd signaal terug ontvangt en bepaalt zo de afstand 
tot een obstakel.\cite{SharpDistanceSensor} De keuze viel op deze sensor omdat die de meest bruikbare 
informatie geeft van alle sensors. Met een lichtsensor is het lastig werken 
zonder lichtbronnen. Een andere optie is het gebruik van sleepcontacten op een geleidende ondergrond (de aluminiumstrip). Een lichtsensor geeft echter meer informatie over de precieze positie van de wagen in zijn omgeving.

Voor de besturing van de rover wordt één servomotor gebruikt die aangesloten is 
op een stuurmechanisme. Aan de hand van een PWM-signaal, gebaseerd op 
pulsbreedtes, kan de servomotor in de gewenste hoek geplaatst worden.

Aangezien een Arduino-microcontroller geen grote stromen kan leveren, moeten signaalstromen en voedingsstromen gesplitst worden. Daarom wordt er gebruik gemaakt van veldeffecttransistoren met drie pinnen (gate, drain en soource). Wij gebruiken n-type MOSFETS. Bij dit type is de stroom tussen drain en source afhankelijk van de spanning tussen gate en source. 
Boven een zekere drempelspanning gaat de MOSFET volledig in geleiding. Op die manier kunnen we de motoren niet enkel aan en uitschakelen, maar ook het vermogen regelen en daarmee het toerental en de uiteindelijke snelheid van de rover.\cite{MosfetBitwizard} 

Om de wagen aan te drijven wordt een MM28 DC-motor gebruikt. Wanneer het een muur te dicht nadert, moet het in staat zijn om achteruit te rijden. In de praktijk komt dit neer op het ompolen van de motoren. Om dit te automatiseren wordt er gebruik gemaakt van een dubbele relais. Waneer er spanning op de spoel in de relais wordt gezet, zullen beide schakelaren in de relais omkeren en zal de positieve pool de negatieve worden en omgekeerd. De relais dissipeert meer stroom dan door de Arduino geleverd kan worden. Daarom wordt ook hier een MOSFET gebruikt om voedings- en signaalstroom van elkaar te isoleren. 

Aan het einde van het parcours zal de afstandssensor zowel vooraan, links als 
rechts het signaal krijgen dat er een obstakel in de weg staat. Dan zal er een 
LED knipperen.
