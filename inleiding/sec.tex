 
\section{Inleiding}
Bij het verkennen van de ruimte en andere hemellichamen zijn er verschillende 
factoren die tot problemen kunnen leiden. De grote afstand is daar één van. Op 
een andere planeet zou een verkenningsrover hierdoor een vertraging van 
inkomende en uitgaande signalen ondervinden. Deze vertraging zou tot problemen 
kunnen leiden bij het besturen van de rover. Dit probleem kan opgelost worden 
door de rover voor een deel autonoom te maken. Deze kan dan zelf een veilige 
weg 
vinden en obstakels ontwijken. Er zijn ook andere zaken waarbij een autonome 
wagen nuttig kan zijn. Google is bijvoorbeeld bezig met het bouwen van een 
wagen 
die zonder bestuurder de weg op kan. Deze zou dan vanzelf de beste en veiligste 
weg naar de bestemming zoeken. Doordat een computer veel sneller kan reageren dan mensen
zou dit het aantal verkeersongelukken drastisch kunnen verlagen. Dit zou ook tot minder files en 
benzineverbruik leiden omdat computers veel efficiënter kunnen rijden dan mensen. 
\cite{Googlecar}

De marsrover Curiosity is een voorbeeld van zo’n autonome rover. Deze bezit 17 
camera's waarmee hij zijn omgeving kan aftasten. Aan de hand van de verzamelde 
informatie kan de Curiosity een veilig pad over het marsoppervlak vinden. Dit 
spaart tijd uit voor de bestuurders van de rover op aarde.\cite{NASACuriosity, 
NASA2013-259} 

De specifieke opdracht bestaat uit het ontwerpen van een wagentje dat autonoom 
een vooraf onbekend parcours moet afleggen. Het parcours is maximaal 12 m lang, 
minimaal 40 cm breed en de afbakenende muren zijn 18 cm hoog. De af te leggen 
weg bestaat uit rechte hoeken en er zijn geen plaatsen waarin het wagentje vast 
kan komen te zitten (er zijn geen doodlopende zijwegen). Het wagentje moet het 
parcours zo snel mogelijk afleggen en bij het bereiken van de finish een 
visueel 
of auditief signaal geven. De totaalprijs van het ontwerp mag maximaal \euro 80 
bedragen.

Met behulp van een afstandssensor verzamelt de wagen informatie over de 
omgeving. De Arduino Uno-controller ontvangt de metingen. Daarna verwerkt deze  
de informatie en zendt dan weer aan de hand van computercode signalen uit naar 
de motoren en het stuurmechanisme zodat de wagen op de juiste baan blijft. Op 
deze manier kan de wagen autonoom een onbekend parcours afleggen, zoals 
de opdracht vereist.

In dit verslag zal in het eerste deel de definitieve conceptkeuze, met name het 
mechanische en elektronische ontwerp aan bod komen. Ook de materiaalkeuze voor 
de rover komt hier aan bod. De volgende sectie “Experimenten” bevat de 
ontwerpberekeningen en de uitgevoerde experimenten. Deze zijn ondersteund met 
informatie uit bijlages A en B. Daarna volgt een bespreking van de resultaten 
van de demonstratie. Als laatste wordt het behalen van de deadlines besproken.
Doorheen het verslag zullen er enkele belangrijke bevindingen aan bod komen. De 
theoretische optimale overbrengingsverhouding kan bijvoorbeeld niet behaald 
worden. De luchtweerstand is pas merkbaar bij hoge snelheden en kan dus 
verwaarloosd worden bij het rijden op het parcours.  
